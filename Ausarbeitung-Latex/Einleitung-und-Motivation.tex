\label{Einleitung-und-Motivation}
In jüngster Zeit haben Universitäten mit immer höher werdenden Abbruchraten zu kämpfen. Diese resultieren natürlich zum einen in frustrierte Studiendekane und Studenten und zum anderen in einen gravierenden Fachkräftemangel, wie wir ihn heutzutage zum Beispiel an der Nachfrage von Ärzten oder auch Informatikern sehen. Um dieses Problem vorzubeugen eignen sich vor allem Self-Assessment-Tests. Diese Tests absolvieren angehende Studenten, um ihre eigenen Interessen und Fähigkeiten besser einschätzen zu können und die bei mangelhafter Leistung gezielt Hilfe anbieten. So können Einbahnstraßen vermieden und der Weg in Richtung Erfolg eingeschlagen werden. Hierbei ist eine eindeutige, einfache Bedienung und vor allem ein persönliches Feedback von hoher Wichtigkeit. Letzteres ermöglicht eine grobe \textit{Studientauglichkeit} und hilft mit Verweisen auf benötigte Grundlagenkurse bei Schwierigkeiten, sodass das Studium mit einem gerechten Niveau angegangen werden kann.
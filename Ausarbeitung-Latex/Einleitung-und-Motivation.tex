\label{Einleitung-und-Motivation}
In jüngster Zeit haben Universitäten mit immer höher werdenden Abbruchraten zu kämpfen. 
Der Grund für den Abbruch des Studiums ist dabei häufig, dass die Studierenden nicht ausreichend über die Inhalte und die Anforderungen ihres Studiengans informiert wurden.
Um diesem Problem vorzubeugen werden von vielen Universitäten sogenannte Self-Assessment-Tests angeboten.

Der Aufwand der Studieninteressenten wird dabei miniert, indem solche Tests online verfügbar sind.
Angehende Studierende sollen durch das Absolvieren des Tests eine Einschätzung ihrer eigenen Fähigkeiten bekommen. 
Ferner soll ihnen verdeutlicht werden, ob der angestrebte Studiengang ihren Vorstellungen und Interessen entspricht.
Hierbei ist eine eindeutige, einfache Bedienung und vor allem ein persönliches Feedback von hoher Wichtigkeit. 

Das Ziel dieser Arbeit ist es die Erstellung eines Online-Selfassessment-Tests zu vereinfachen oder es für einen Nichtprogrammierer überhaupt möglich zu machen. 
Zu diesem Zweck wurde unser Testgenerator mit einer einfach zu bedienenden Benutzeroberfläche entwickelt.
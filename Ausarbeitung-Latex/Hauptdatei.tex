% allgem. Dokumentenformat
\documentclass[a4paper,12pt,headsepline,twocolumn]{scrartcl}

% weitere Pakete
% Grafiken aus PNG Dateien einbinden
\usepackage{graphicx}

%deletes first blank page
\usepackage{atbegshi}% http://ctan.org/pkg/atbegshi
\AtBeginDocument{\AtBeginShipoutNext{\AtBeginShipoutDiscard}}

% Deutsche Sonderzeichen benutzen 
%\usepackage{ngerman}

% deutsche Silbentrennung
\usepackage[ngerman]{babel}

% Eurozeichen einbinden
\usepackage[right]{eurosym}

% Umlaute unter UTF8 nutzen
\usepackage[utf8]{inputenc}

% Zeichenencoding
\usepackage[T1]{fontenc}

\usepackage{lmodern}
\usepackage{fix-cm}

% floatende Bilder ermöglichen
%\usepackage{floatflt}

% mehrseitige Tabellen ermöglichen
\usepackage{longtable}

% Unterstützung für Schriftarten
%\newcommand{\changefont}[3]{ 
%\fontfamily{#1} \fontseries{#2} \fontshape{#3} \selectfont}

% Packet für Seitenrandabständex und Einstellung für Seitenränder
\usepackage{geometry}
\geometry{left=3.5cm, right=2cm, top=2.5cm, bottom=2cm}

% Paket für Boxen im Text
\usepackage{fancybox}

% bricht lange URLs "schoen" um
\usepackage[hyphens,obeyspaces,spaces]{url}

% Paket für Textfarben
\usepackage{color}

% Mathematische Symbole importieren
\usepackage{amssymb}

% neue Kopfzeilen mit fancypaket
\usepackage{fancyhdr} %Paket laden
\pagestyle{fancy} %eigener Seitenstil
\fancyhf{} %alle Kopf- und Fußzeilenfelder bereinigen
\fancyhead[L]{\nouppercase{\leftmark}} %Kopfzeile links
\fancyhead[C]{} %zentrierte Kopfzeile
\fancyhead[R]{\thepage} %Kopfzeile rechts
\renewcommand{\headrulewidth}{0.4pt} %obere Trennlinie
%\fancyfoot[C]{\thepage} %Seitennummer
%\renewcommand{\footrulewidth}{0.4pt} %untere Trennlinie

% für Tabellen
\usepackage{array}

% Runde Klammern für Zitate
%\usepackage[numbers,round]{natbib}

% Festlegung Art der Zitierung - Havardmethode: Abkuerzung Autor + Jahr
\bibliographystyle{alphadin}

% Schaltet den zusätzlichen Zwischenraum ab, den LaTeX normalerweise nach einem Satzzeichen einfügt.
\frenchspacing

% Paket für Zeilenabstand
\usepackage{setspace}

% für Bildbezeichner
\usepackage{capt-of}
\makeindex
\begin{document}
	
% Leere Seite am Anfang
\newpage
\thispagestyle{empty} % erzeugt Seite ohne Kopf- / Fusszeile
\section*{ }

%Titelseite

\begin{titlepage}
		
		\centering
		\includegraphics[width=0.55\textwidth]{unistuttgart_logo_deutsch}\par\vspace{1cm}
		{\scshape\Large Bachelorforschungsprojekt Informatik\par}
		\vspace{1.5cm}
		{\huge\bfseries Entwicklung eines
			Online-Self-Assessment-Tests für
			Studieninteressenten der Informatik\par}
		\vspace{2cm}
		{\Large\itshape Jonas Allali\par}
		\vspace{0.2cm}
		{\Large\itshape Julian Blumenröther\par}
		\vspace{0.2cm}
		{\Large\itshape Tim-Julian Ehret\par}
		\vspace{0.2cm}
		{\Large\itshape Sokol Makolli\par}
		\vspace{0.2cm}
		{\Large\itshape Jena Satkunarajan\par}
		\vspace{1cm}
		{\Large Prüfer: \par
			\vspace{0.3cm}
		\textsc{Prof. Dr.-Ing. Stefan Funke}}
		
		\vfill
		
		% Bottom of the page
		{\large \today\par}
\end{titlepage}	
\tableofcontents
% inkludiere einzelne Abschnitte
\clearpage
\section{Abstract}
In dieser Arbeit wurde ein webunterstützender Self-Assessment-Test in Java entwickelt. Ersteller sind in der Lage ohne HTML-Kenntnisse Fragen hinzuzufügen. Das Programm generiert dann einen funktionsfähigen Online-Test bei gegebenem Server. Diese Arbeit fasst die genaue Entwicklung des Tests zusammen, indem zuerst nach einer Motivation auf andere Online-Tests eingegangen wird. Daraufhin wird die Architektur des Projekts präsentiert, die sich in GUI, Parser, Generator und Webseite aufteilen lässt. Gegen Ende der Arbeit wird ein Anwendungsszenario gezeigt, das mit einem zusammenfassenden Ausblick abgerundet wird.
\section{Einleitung und Motivation}
In jüngster Zeit haben Universitäten mit immer höher werdenden Abbruchraten zu kämpfen. Diese resultieren natürlich zum einen in frustrierte Studiendekane und Studenten und zum anderen in einen gravierenden Fachkräftemangel, wie wir ihn heutzutage zum Beispiel an der Nachfrage von Ärzten oder auch Informatikern sehen. Um dieses Problem vorzubeugen eignen sich vor allem Self-Assessment-Tests. Diese Tests absolvieren angehende Studenten, um ihre eigenen Interessen und Fähigkeiten besser einschätzen zu können und die bei mangelhafter Leistung gezielt Hilfe anbieten. So können Einbahnstraßen vermieden und der Weg in Richtung Erfolg eingeschlagen werden. Hierbei ist eine eindeutige, einfache Bedienung und vor allem ein persönliches Feedback von hoher Wichtigkeit. Letzteres ermöglicht eine grobe \textit{Studientauglichkeit} und hilft mit Verweisen auf benötigte Grundlagenkurse bei Schwierigkeiten, sodass das Studium mit einem gerechten Niveau angegangen werden kann.
\section{Related Work}
\label{Jonas}
Um sich einen Überblick über die Thematik verschaffen zu können, wurden verschiedene Self-Assessment-Tests anderer Universitäten analysiert.
Hierbei ist wichtig zu wissen, dass folgende Beschreibungen sich nur auf die jeweiligen Benutzeroberflächen der Tests beziehen.

\subsection{Test der Universität Frankfurt}
Der Test
\footnote{\url{https://www.gdv.informatik.uni-frankfurt.de/selfassessment/Informatik/}} 
der Universität Frankfurt beginnt mit einem Motivationstext, der sowohl Sinn und Zweck des Tests erklärt, als auch den User in die Benutzung einführt.

Desweiteren wird angeboten den Test der Universität Frankfurt herunterzuladen, was eine Offline-Bearbeitung realisiert. 
Bei unserem Test wird der aktuelle Zustand in der URL der Webseite kodiert. 
Dies ermöglicht zwar keine Bearbeitung ohne Internet, bietet aber an, den Test jederzeit zu pausieren, indem man sich die URL abspeichert.

Der zu vergleichende Test erfordert außerdem eine persönliche Registrierung, welche sehr ausführlich ist, was dazu führt, dass die Benutzererfahrer sinkt. Daher wurde eine Registrierung in unserem Test weggelassen, um die Flexibilität und Einfachheit zu gewährleisten.  
\begin{figure*} 
  \centering
     \includegraphics[width=\textwidth]{Jonas_Images/frankfurt1.png}
  \caption{}
  \label{fig:Bild1}
\end{figure*}
In Abbildung~\ref{fig:Bild1} ist das Grundlayout des Tests von der Universität Frankfurt zu sehen. 
Das Layout ist sehr ähnlich zu unserem und beinhaltet die Frage mit ihren Antworten, einen Fortschrittsbalken, einen Next-Button und eine Zeitanzeige.

Auch ist es möglich Grafiken anzeigen zu lassen.
Der Test wird in verschiedene Kategorien eingegliedert, die wiederholbar sind. 
Bei unserem Test ist die Erstellung von Kategorien ebenfalls möglich, ein Mehrfachbeantwortung ist jedoch ausgeschlossen.

Abschließend bietet der Test der Universität Frankfurt eine Bewertung der beantworteten Fragen.
Anders als die Universität Frankfurt enthält unser Test insbesondere eine persönliche Beurteilung des Erstellers.
Diese diehnt als Abschließendes Feedback für die erbrachte Leistung des Nutzers im Test.

\subsection{Test der RWTH Aachen}
\begin{figure*}[htbp] 
  \centering
     \includegraphics[width=0.5\textwidth]{Jonas_Images/Abschnitte.png}
  \caption{}
  \label{fig:Bild4}
\end{figure*}
Der Test der RWTH Aachen \footnote{\url{https://www.global-assess.rwth-aachen.de/rwth/tm_alt/}} ähnelt unserem Ansatz, wie man in Abbildung \ref{fig:Bild4} sehen kann.
Das Layout ist einfach gehalten und überschaubar. 
Es gibt einen Next-Button und einen Fortschrittsbalken.
In beiden Ansätzen besteht keine Möglichkeit, eine Frage zu überspringen.
Außerdem bedarf es einer Anmeldung, um am Test der Hochschule Aachen teilnehmen zu können.
Damit die Hemmschwelle zur Teilnahme möglichst niedrig gehalten wird, haben wir dafür entschieden, auf jede Form der Anmeldung zu verzichten.






\section{Architektur}

\subsection{GUI}
\section{Julian's Abschnitt}\label{Julian}
\subsection{Hier kommt der erste Teil-Abschnitt }
bliBlaBlub


\subsection{Parser}
\label{Tim}
In diesem Kapitel wird die Parser Komponente vorgestellt. Ihre primäre Aufgabe besteht darin, zwischen dem Speichermedium und dem Generierungsprozess zu vermitteln. Als Speichermedium dienen hierbei einfache XML Dateien, während die aktive Generierung mit Java Objekten arbeitet.\\
Im folgenden wird zunächst die Interne Struktur vorgestellt, welche der Implementierung zugrunde liegt. Anschließend wird die Funktionalitäten der Parser Komponente im Detail vorgestellt. 

\subsection{Interne Struktur}
Generell sind alle hier relevanten Teile eines Self Assessment Tests ( Kategorien, Fragen, Antworten, Ergebnis ) Java Objekte. 
Nachdem eine Frage erstellt wurde ist sie zunächst ein Java Objekt der Klasse 'Question'. Als solches besitzt die Frage verschiedene Attribute wie z.b. ihre Kategorie, den Aufgabentext und insbesondere eine Liste der ihr zugehörigen Antworten. Somit ist gleichzeitig das Mapping zwischen Fragen und Kategorien, sowie zwischen Fragen und Antworten gegeben. Um den ganzen erstellten Test zu sichern genügt es also, eine Liste der Fragen speichern. Der Einfachheit halber haben wir eine Toplevel Klasse namens 'SARoot' eingeführt. Ein Test wird so letztlich durch ein Objekt der Klasse 'SARoot' repräsentiert, welches eine Liste von Fragen und Kategorien, so wie das Ergebnis beinhaltet.   

\subsection{Speichern von Fragen}
Wie in Kapitel 4.2 beschrieben, kann der Ersteller den aktuellen Arbeitsstand speichern, indem er den Test als XML Datei exportiert. In der GUI findet man diese Operation unter 'Export XML'.
Hierbei werden im Hintergrund alle erstellten Java Objekte an die Parser Komponente übergeben, welche diese dann in XML Elemente umwandelt und in einer Datei abspeichert. Genauer gesagt wird hier genau ein Toplevel Objekt übergeben, welches alle Objekte beinhaltet.\\
Intern verwendet die Parser Komponente die 'Java Architecture for XML Binding (JAXB)'\cite{JAXB}. Der von JAXB\cite{JAXB} bereitgestellte 'marshaller' ermöglicht es Java Objekte von spezifizierten Klassen direkt in XML Elemente zu verwandeln und anschließend in eine XML Datei zu schreiben. Um vom 'marshaller' erkannt zu werden benötigen alle unsere Java Klassen die von JAXB\cite{JAXB} vorgeschlagenen XML Tags.\\
Die Parser Komponente nutzt diese Funktion der JAXB\cite{JAXB} API um unser Speichermedium, eine XML Datei, zu erstellen. So können die nur zur Laufzeit existierenden Objekte persistent gespeichert werden.

\subsection{Einlesen von Fragen}
Um gespeicherte Fragen oder ganze Tests wiederverwendbar zu machen, bietet die Parser Komponente die Möglichkeit XML Dateien einzulesen. Es findet als eine Überführung von XML Elementen in Java Objekte statt. Ähnlich wie beim Speichern verwendet die Parser Komponente hierzu wieder eine Funktionalität des JAXB\cite{JAXB} 'marshaller's. Die Funktionalität besteht darin, aus XML Elementen Java Objekte von Klassen mit passenden XML Tags zu erzeugen.\\
Das Resultat des Einlesens ist ein Toplevel Objekt der Klasse 'SARoot'. Aus diesem Objekt kann der gesamte Self Assessment Test erstellt und die Webseite generiert werden. \\
Die GUI greift auf die Parser Komponente zurück um die 'Import XML' Operation durchzuführen. Außerdem ist es dem erfahrenen Ersteller nun möglich den gesamten Self Assessment Test 'von Hand' in einer XML Datei zu verfassen, ohne hierfür die GUI zu verwenden. 






\subsection{Generator}
\label{Sokol}
Für das einfache Verwalten der Webseite, ist es vorgesehen, dass sie statisch ist.
In diesem Fall heißt das, dass es keinen Server geben soll, der dynamisch auf Anfragen des Benutzers reagiert.
Alle dynamischen Funktionen, wie das Laden neuer Fragen, finden auf der Benutzerebene statt.
Der Server liefert dem Benutzer bei Bedarf, also nur statische Dateien.

Der Vorteil dieses Designs ist, dass die Webseitendateien nur ein Mal aus den Java Objekten generiert werden müssen.
Für diese Funktion haben wir uns für die Template Engine Velocity\footnote{\url{http://velocity.apache.org/}} entschieden.

\subsubsection{Velocity}

Velocity erlaubt es Dokumente mit Variablen zu bestücken, die dann von Velocity mit dem Text aus den Java Objekten ersetzt werden.
Diese Dokumente werden Templates genannt.
Dafür werden Velocity das Java Objekt, der Name des Java Objektes in dem Tamplate und das Template an sich übergeben.
Velocity liest daraufhin das Template, sucht sich die Stellen heraus, die Variablen enthalten, und ersetzt diese mit den Inhalten des Java Objektes.

Ein Beispiel eines Velocity Templates ist in Listing~\ref{lst:velocity-example} gegeben.
Für das Beispiel wird angenommen, dass Velocity ein Question Objekt übergeben bekommt.
Dieses Question Objekt hat eine Funktion 'getContent()', die einen String zurückgibt, und eine andere Funktion 'getAnswers()', die eine Liste mit Answer Objekten zurückgibt.
Das Answer Objekt hat wiederum auch eine Funktion 'getContent()'.
Außerdem wird Velocity der Variablenname 'question' und das aufgeführte Template übergeben.

Das Zeichen '\$' in dem Template signalisiert Velocity, dass der danach kommende Text für Velocity vorgesehen ist.
So bemerkt Velocity, dass '\$question', das übergebene Question Objekt referenziert und ruft im ersten Fall die Funktion 'getContent()' auf.
Die Funktion wird ausgewertet und der zurückgegebene String wird an der Stelle des Funktionsaufrufs gesetzt.

Im Beispiel-Template sieht man auch eine 'foreach' Schleife, die mit '\#foreach' beginnt und mit '\#end' endet.
Diese Schleife sorgt dafür, dass der Abschnitt, der sich in der Schleife befindet, so oft geschrieben wird, wie es Answer Objekte in der von 'question.getAnswers()' zurückgegebenen Liste gibt.

Daraufhin wird '\$answer.getContent()' mit der entsprechenden Rückgabe der Answer Objektes ersetzt.


\begin{lstlisting}[basicstyle=\tiny,label={lst:velocity-example},caption={Beispiel eines Velocity Templates.},language=HTML]
<h1>$question.getContent()</h1>
<ul>
#foreach($answer in $question.getAnswers())
<li>$answer.getContent()</li>
#end
</ul>
\end{lstlisting}

\subsubsection{Erstellung der Webseite}
Für alle Dateien der Webseite, die Inhalte von den Java Objekten benötigen, wird eine Template Datei erstellt.
Daraufhin werden die ausgewerteten Templates, mit den anderen für die Webseite benötigten Dateien, in ein ZIP-Archiv gepackt und in einen von dem Benutzer des Generators festgelegten Ort gespeichert.

Um die Webseite dann zu veröffentlichen, müssen die Dateien im Archiv über einen HTTP-Server angeboten werden.





\subsection{Website}
\section{Jena's Abschnitt}\label{Jena}
\subsection{Hier kommt der erste Teil-Abschnitt }
bliBlaBlub


\section{Anwendungsszenario}

\section{Zusammenfassung und Ausblick}

\onecolumn
% einfacher Zeilenabstand
\singlespacing
% Literaturliste soll im Inhaltsverzeichnis auftauchen
\newpage
\addcontentsline{toc}{section}{Literaturverzeichnis}
% Literaturverzeichnis anzeigen
\renewcommand\refname{Literaturverzeichnis}

\bibliography{Hauptdatei}
\end{document}
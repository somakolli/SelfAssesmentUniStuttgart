\label{Sokol}
Für das einfache Verwalten der Webseite, ist es vorgesehen, dass sie statisch ist.
In diesem Fall heißt das, dass es keinen Server geben soll, der dynamisch auf Anfragen des Benutzers reagiert.
Alle dynamischen Funktionen, wie das Laden neuer Fragen, finden auf der Benutzerebene statt.
Der Server liefert dem Benutzer bei Bedarf, also nur statische Dateien.

Der Vorteil dieses Designs ist, dass die Webseitendateien nur ein Mal aus den Java Objekten generiert werden müssen.
Für diese Funktion haben wir uns für die Template Engine \href{http://velocity.apache.org/}{Velocity} entschieden.

\subsubsection{Velocity}

Velocity erlaubt es Dokumente mit Variablen zu bestücken, die dann von Velocity mit dem Text aus den Java Objekten ersetzt werden.
Diese Dokumente werden Templates genannt.
Dafür werden Velocity das Java Objekt, der Name des Java Objektes in dem Tamplate und das Template an sich übergeben.
Velocity liest daraufhin das Template, sucht sich die Stellen heraus, die Variablen enthalten, und ersetzt diese mit den Inhalten des Java Objektes.

Ein Beispiel eines Velocity Templates ist in Listing~\ref{lst:velocity-example} gegeben.
Für das Beispiel wird angenommen, dass Velocity ein Question Objekt übergeben bekommt.
Dieses Question Objekt hat eine Funktion 'getContent()', die einen String zurückgibt, und eine andere Funktion 'getAnswers()', die eine Liste mit Answer Objekten zurückgibt.
Das Answer Objekt hat wiederum auch eine Funktion 'getContent()'.
Außerdem wird Velocity der Variablenname 'question' und das aufgeführte Template übergeben.

Das Zeichen '\$' in dem Template signalisiert Velocity, dass der danach kommende Text für Velocity vorgesehen ist.
So bemerkt Velocity, dass '\$question', das übergebene Question Objekt referenziert und ruft im ersten Fall die Funktion 'getContent()' auf.
Die Funktion wird ausgewertet und der zurückgegebene String wird an der Stelle des Funktionsaufrufs gesetzt.

Im Beispiel-Template sieht man auch eine 'foreach' Schleife, die mit '\#foreach' beginnt und mit '\#end' endet.
Diese Schleife sorgt dafür, dass der Abschnitt, der sich in der Schleife befindet, so oft geschrieben wird, wie es Answer Objekte in der von 'question.getAnswers()' zurückgegebenen Liste gibt.

Daraufhin wird '\$answer.getContent()' mit der entsprechenden Rückgabe der Answer Objektes ersetzt.


\begin{lstlisting}[basicstyle=\tiny,label={lst:velocity-example},caption={Beispiel eines Velocity Templates.},language=HTML]
<h1>$question.getContent()</h1>
<ul>
#foreach($answer in $question.getAnswers())
<li>$answer.getContent()</li>
#end
</ul>
\end{lstlisting}

\subsubsection{Erstellung der Webseite}
Für alle Dateien der Webseite, die Inhalte von den Java Objekten benötigen, wird eine Template Datei erstellt.
Daraufhin werden die ausgewerteten Templates, mit den anderen für die Webseite benötigten Dateien, in ein ZIP-Archiv gepackt und in einen von dem Benutzer des Generators festgelegten Ort gespeichert.

Um die Webseite dann zu veröffentlichen, müssen die Dateien im Archiv über einen HTTP-Server angeboten werden.



\documentclass[12pt,pdftex,a4paper]{article}
\usepackage[ngerman]{babel}
\usepackage{amsmath}
\usepackage{amssymb}
\usepackage{cancel}
\usepackage[T1]{fontenc} 
\usepackage[utf8]{inputenc} 
\usepackage{listings}
\usepackage{gauss} 
\usepackage{amsmath}
\usepackage{underscore}
\newcommand{\bbN}{\mathbbm{N}}
\newcommand{\bbR}{\mathbbm{R}}
\newcommand{\bbZ}{\mathbbm{Z}}
\newcommand{\bbI}{\mathbbm{I}}
\usepackage[pdftex]{graphicx}
\usepackage{listings}
\usepackage{pgf}
\usepackage{csvsimple}
\lstset{language=Python,basicstyle=\footnotesize}
\begin{document}
\title{Self Assesment Tests für die Universität Stuttgart - Anforderungen}
\author{Tim-Julian Ehret, Sokol Makolli, Julian Blumenröther, Jena, Jonas Allali}
\maketitle
%%%%%%%%%%%%%%%%%%%%%%%%%%%%%%%%%%%%%%%%%%%%%%%%
\section{Funktionale Anforderungen}
\subsection{Fragenaufbau}
\begin{itemize}
 \item Zunächst sollen lediglich Single-Choice und Multiple-Choice Fragen unterstützt werden.
 \item Die Lösung soll weder am Anfang noch am Ende angezeigt werden.
 \item Man soll Bilder sowie Videos in den Fragen und in den Antworten einbinden können.
 \item Der Fragensteller soll Zeitlimits einstellen können; zunächst nur für jede Frage.
\end{itemize}
\subsection{Fragenbewertung}
\begin{itemize}
 \item Man soll die Fragen gewichten können.
 \item Am Ende des Tests soll der Anteil der richtig Beantworteten Fragen in Prozent angegeben werden.
 \item Je nach Ergebnis soll ein Text angezeigt werden den der Fragensteller bestimmen kann.
\end{itemize}
\subsection{Darstellung}
\begin{itemize}
 \item Es soll ein Fortschrittsbalken angezeigt werden, jedoch soll man nicht zwischen den Fragen hin und her springen können.
 \item Die Fragen sollen in Abschnitte unterteilt werden.
 \item Keine unterschtützung für mathematische Formeln.
\end{itemize}
\subsection{State-Management}
\begin{itemize}
 \item Kein Server nur statische Seiten!
 \item Keine Cookies!
 \item Der State(also die beantworteten Fragen) sollen im Link dargestellt werden.
\end{itemize}
\section{Nicht-Funktionale Anforderungen}
\begin{itemize}
 \item Ein Benutzerhandbuch für die Fragensteller soll erstellt werden.
\end{itemize}
\end{document}
\grid
\grid
